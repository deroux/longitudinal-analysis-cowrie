\documentclass[ngerman,noconfig]{uibklttr}

% options are passed to the underlying scrlttr2 class
% e.g., to set language to English use:
% \documentclass[english]{uibklttr}
%
% to read your personal configuration (name, faculty, department, ...)
% from uibklttr.cfg omit the 'noconfig' option
% \documentclass[ngerman]{uibklttr}

\begin{document}

% use KOMA variables to specify your personal details (name, faculty, department, ...)
% either these can be stored in a uibklttr.cfg (see 'noconfig' above) or specified directly
\setkomavar{subject}{Mustervorlage}
\setkomavar{signature}{Dr.~Max Mustermann\\Funktion}

% % further KOMA variables include:
% % header
% \setkomavar{subtext}{Fakultät für\\Bildungswissenschaften}
% \setkomavar{institute}{Institut für Erziehungswissenschaft\\Dr.~Max Mustermann}
%
% % reference line
% \setkomavar{fromname}[Name]{Dr.~Max Mustermann}
% \setkomavar{fromemail}{Max.Mustermann@uibk.ac.at}
% \setkomavar{fromphone}{+43 512 507-0000}
% \setkomavar{fromfax}{+43 512 507-0000}
% \setkomavar{refnum}{123456-789}
%
% % footer
% \setkomavar{fromaddress}{Universität Innsbruck, Christoph-Probst-Platz-Innrain 52, 6020 Innsbruck, Austria}


\begin{letter}[fromemail]{%
  Frau\\
  Univ.-Prof.~Gabriele Mustermann\\
  Musterstraße 123\\
  6020 Innsbruck, Austria}

\opening{Sehr geehrte Damen und Herren,}

Dies ist ein Typoblindtext. An ihm kann man sehen, ob alle Buchstaben da sind
und wie sie aussehen. Manchmal benutzt man Worte wie Hamburgefonts, Rafgenduks
oder Handgloves, um Schriften zu testen. Manchmal Sätze, die alle Buchstaben des
Alphabets enthalten - man nennt diese Sätze »Pangrams«. Sehr bekannt ist dieser:
The quick brown fox jumps over the lazy old dog. Oft werden in Typoblindtexte
auch fremdsprachige Satzteile eingebaut (AVAIL® and WefoxTM are testing aussi la
Kerning), um die Wirkung in anderen Sprachen zu testen.

In Lateinisch sieht zum Beispiel fast jede Schrift gut aus. Quod erat
demonstrandum. Seit 1975 fehlen in den meisten Testtexten die Zahlen, weswegen
nach TypoGb. 204 § ab dem Jahr 2034 Zahlen in 86 der Texte zur Pflicht werden.
Nichteinhal- tung wird mit bis zu 245 €oder 368 \$ bestraft. Genauso wichtig in
sind mittlerweile auch Âçcèñtë, die in neueren Schriften aber fast immer
enthalten sind. Ein wichtiges aber schwierig zu integrierendes Feld sind
OpenType-Funktionalitäten. Je nach Software und Voreinstellungen können
eingebaute Kapitälchen, Kerning oder Ligaturen (sehr pfiffig) nicht richtig
dargestellt werden.Dies ist ein Typoblindtext. An ihm kann man sehen, ob alle
Buchstaben da sind und wie sie aussehen.  Manchmal benutzt man Worte wie
Hamburgefonts, Rafgenduks oder Handgloves, um Schriften zu testen.

Lateinisch sieht zum Beispiel fast jede Schrift gut aus. Quod erat
demonstrandum. Seit 1975 fehlen in den meisten Testtexten die Zahlen, weswegen
nach TypoGb. 204 § ab dem Jahr 2034 Zahlen in 86 der Texte zur Pflicht werden.
Nichteinhaltung wird mit bis zu 245 oder 368 \$ bestraft. Genauso wichtig in
sind mittlerweile auch Âçcèñtë, die in neueren Schriften aber fast immer
enthalten sind. Ein wichtiges aber schwierig zu integrierendes Feld sind
OpenType-Funktionalitäten. Je nach Software und Voreinstellungen können
eingebaute Kapitälchen, Kerning oder Ligaturen (sehr pfiffig) nicht richtig
dargestellt werden. Dies ist ein Typoblindtext. An ihm kann man sehen, ob alle
Buchstaben da sind und wie sie aussehen.

\closing{Mit freundlichen Grüßen}
%\ps
%PS:
%\encl{etwaige Anlagen}
%\cc{zusätzliche Empfänger}

\clearpage

Dies ist ein Typoblindtext. An ihm kann man sehen, ob alle Buchstaben da sind
und wie sie aussehen. Manchmal benutzt man Worte wie Hamburgefonts, Rafgenduks
oder Handgloves, um Schriften zu testen. Manchmal Sätze, die alle Buchstaben des
Alphabets enthalten - man nennt diese Sätze »Pangrams«. Sehr bekannt ist dieser:
The quick brown fox jumps over the lazy old dog. Oft werden in Typoblindtexte
auch fremdsprachige Satzteile eingebaut (AVAIL® and Wefox™ are testing aussi la
Kerning), um die Wirkung in anderen Sprachen zu testen. In Lateinisch sieht zum
Beispiel fast jede Schrift gut aus. Quod erat demonstrandum. Seit 1975 fehlen in
den meisten Testtexten die Zahlen, weswegen nach TypoGb. 204 § ab dem Jahr 2034
Zahlen in 86 der Texte zur Pflicht werden. Nichteinhaltung wird mit bis zu 245 €
oder 368 \$ bestraft.  Genauso wichtig in sind mittlerweile auch Âçcèñtë, die in
neueren Schriften aber fast immer enthalten sind. Ein wichtiges aber schwierig
zu integrierendes Feld sind OpenType-Funktionalitäten. Je nach Software und
Voreinstellungen können eingebaute Kapitälchen, Kerning oder Ligaturen (sehr
pfiffig) nicht richtig dargestellt werden.  Dies ist ein Typoblindtext. An ihm
kann man sehen, ob alle Buchstaben da sind und wie sie aussehen.  Dies ist ein
Typoblindtext. An ihm kann man sehen, ob alle Buchstaben da sind und wie sie
aussehen. Manchmal benutzt man Worte wie Hamburgefonts, Rafgenduks oder
Handgloves, um Schriften zu testen. Manchmal Sätze, die alle Buchstaben des
Alphabets enthalten - man nennt diese Sätze »Pangrams«. Sehr bekannt ist dieser:
The quick brown fox jumps over the lazy old dog. Oft werden in Typoblindtexte
auch fremdsprachige Satzteile eingebaut (AVAIL® and Wefox™ are testing aussi la
Kerning), um die Wirkung in anderen Sprachen zu testen. In Lateinisch sieht zum
Beispiel fast jede Schrift gut aus. Quod erat demonstrandum. Seit 1975 fehlen in
den meisten Testtexten die Zahlen, weswegen nach TypoGb. 204 § ab dem Jahr 2034
Zahlen in 86 der Texte zur Pflicht werden. Nichteinhaltung wird mit bis zu 245 €
oder 368 \$ bestraft.  Genauso wichtig in sind mittlerweile auch Âçcèñtë, die in
neueren Schriften aber fast immer enthalten sind. Ein wichtiges aber schwierig
zu integrierendes Feld sind OpenType-Funktionalitäten. Je nach Software und
Voreinstellungen können eingebaute Kapitälchen, Kerning oder Ligaturen (sehr
pfiffig) nicht richtig dargestellt werden.  Dies ist ein Typoblindtext. An ihm
kann man sehen, ob alle Buchstaben da sind und wie sie aussehen. Dies ist ein
Typoblindtext. An ihm kann man sehen, ob alle Buchstaben da sind und wie sie
aussehen. Manchmal benutzt man Worte wie Hamburgefonts, Rafgenduks oder
Handgloves, um Schriften zu testen. Manchmal Sätze, die alle Buchstaben des
Alphabets enthalten - man nennt diese Sätze »Pangrams«. Sehr bekannt ist dieser:
The quick brown fox jumps over the lazy old dog. Oft werden in Typoblindtexte
auch fremdsprachige Satzteile eingebaut (AVAIL® and Wefox™ are testing aussi la
Kerning), um die Wirkung in anderen Sprachen zu testen. In Lateinisch sieht zum
Beispiel fast jede Schrift gut aus. Quod erat demonstrandum. Seit 1975 fehlen in
den meisten Testtexten die Zahlen, weswegen nach TypoGb. 204 § ab dem Jahr 2034
Zahlen in 86 der Texte zur Pflicht werden. Nichteinhaltung wird mit bis zu 245 €
oder 368 \$ bestraft.  Genauso wichtig in sind mittlerweile auch Âçcèñtë, die in
neueren Schriften aber fast immer enthalten sind. Ein wichtiges aber schwierig
zu integrierendes Feld sind OpenType-Funktionalitäten.
\end{letter}

\end{document}

