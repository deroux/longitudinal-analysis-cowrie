% -------------------------------------------------------------------
% - NAME:        poster.tex
% - AUTHOR:      Reto Stauffer
% - BASED ON:    Jakob Messners version of this theme
% - DATE:        2014-09-29
% -------------------------------------------------------------------
% - DESCRIPTION: This is a demo template for portrait beamer poster
%                in the UIBK design 2017.
% -------------------------------------------------------------------
\documentclass[final]{beamer} 

%\usepackage[orientation=portrait,size=a0,scale=1.30]{beamerposter}
%\usetheme[ncols=2]{uibkposter}
%% ------------------------------------------------------------------
%% Use the two lines above for portrait posters
%% ------------------------------------------------------------------

\usepackage[orientation=landscape,size=a0,scale=1.30]{beamerposter}
\usetheme[ncols=3,orangetheme]{uibkposter}
%% ------------------------------------------------------------------
%% Use the two lines above for landscape posters
%% The option 'orangetheme' can be used to switch the theme.
%% Additional options allowed:
%% - orangetheme: uses alternative color theme
%% ------------------------------------------------------------------

\headerimage{1}
%% ------------------------------------------------------------------
%% The theme offers four different header images based on the
%% corporate design of the university of innsbruck. Currently
%% 1, 2, 3 and 4 is allowed as input to \headerimage{...}. Default
%% or fallback is '1'.
%% ------------------------------------------------------------------

%% ------------------------------------------------------------------
%% The official corporate colors of the university are predefined and
%% can be used for e.g., highlighting something. Simply use
%% \color{uibkorange} or \begin{color}{uibkorange} ... \end{color}
%% Defined colors are:
%% - uibkorange, uibkblue, uibkgray, uibkgraym
%% Please note that there are two faculty colors (see definition above)
%% - uibkcol, uibkcoll
%% The frametitle color can be easily adjusted e.g., to black with
%% \setbeamercolor{titlelike}{fg=black}
%% ------------------------------------------------------------------

%\setbeamercolor{verbcolor}{fg=uibkorange}
%% ------------------------------------------------------------------
%% Setting a highlight color for verbatim output such as from
%% the commands \pkg, \email, \file, \dataset 
%% ------------------------------------------------------------------

%% The title of your poster
\title{Dynamical Statistical Forecast of Alpine Snow Amounts}

%% If the subtitle is not set or empty no subtitle will be shown
\subtitle{This here is a subtitle which I never used in the poster theme.}

%% Author(s) of the poster
\author{Max Mustermann (\textit{max.mustermann@uibk.ac.at}),
   F.C. Demoman and A. Notherone} 

%% Enable numbered captions (figures, tables)
\setbeamertemplate{caption}[numbered]

\usepackage{tikz} %% For the example figure

%% Begin document
\begin{document}

\begin{frame}[fragile]
\begin{columns}[t]

%% ------------------------------------------------------------------
%% This template contains content for both, the portrait version (two
%% columns) and the landscape version (three columns).
%% If \usepackage[orientation=portrait,...]{beamerposter} then the
%% center column is unused (only left and right column). In this
%% case simply remove the block:
%%     \if@beamerposter@portrait
%%         [...]
%%     \fi
%% If \usepackage[orientation=landscape]{beamerposter} the layout 
%% comes with three columns (left, center, right). In this case
%% remove the \if@beamerposter@portrait, \else and \fi condition but
%% keep the block:
%%     \begin{centercolumn}
%%         [...]
%%     \end{centercolumn}
%% ------------------------------------------------------------------

%% ------------------------------------------------------------------
%% begin left column for both, portrait and landscape
%% ------------------------------------------------------------------
\begin{leftcolumn}
   %% please leave one blank line here

   %% first block
   \begin{boxblock}{Introduction}
      This is just a {\bf demo of the \LaTeX} beamer
      poster template for {\bf portrait} posters.
      Conference posters in {\bf portrait} come
      with two columns (left and right) to present
      {\bf your results and findings}.
      This demo {\bf shows some of the features} which
      are available in this \LaTeX theme.

      \vspace{1em}
      Lorem ipsum dolor sit amet, consetetur sadipscing elitr, sed diam nonumy
      eirmod tempor invidunt ut labore et dolore magna aliquyam erat, sed diam
      voluptua. At vero eos et accusam et justo duo dolores et ea rebum. Stet clita
      kasd gubergren, no sea takimata sanctus est Lorem ipsum dolor sit amet.
   \end{boxblock}

   %% second block
   \begin{boxblock}{Bullet Lists Provided}

      The {\bf poster theme} also provides styling for bullet
      point lists.

      \begin{itemize}
         \item Level 1
         \begin{itemize}
            \item Level 2
            \begin{itemize}
               \item Level 3
            \end{itemize}
         \end{itemize}
      \end{itemize}

      \begin{itemize}
         \item Colors can also be used for items in lists \dots
         \color{uibkblue}
         \item such as this one \dots
         \color{uibkorange}
         \item or this one.
      \end{itemize}

      Nam liber tempor cum soluta nobis eleifend option congue nihil imperdiet
      doming id quod mazim placerat facer possim assum. Lorem ipsum dolor sit amet,
      consectetuer adipiscing elit.

      \begin{itemize}
         \item Laoreet dolore magna aliquam erat volutpat.
         \item Ut wisi enim ad minim veniam, quis
         \item Nostrud exerci tation ullamcorper:
         \begin{itemize}
            \item suscipit lobortis nisl
            \item ut aliquip ex ea
         \end{itemize}
      \end{itemize}
   \end{boxblock}

   %% Third block
   \begin{boxblock}{Methodology}

      The equation below is the famous {\bf Drake equation} to 
      illustrate how \LaTeX equations look within this poster theme.
      $N$ is the estimated number of active and communicative extraterrestrial
      civilizations in the Milky Way and can be expressed as follows:

      \begin{equation}
         N = R_* \cdot f_p \cdot n_e \cdot f_l \cdot f_i \cdot f_c \cdot L
      \end{equation}

      \begin{footnotesize}
      \begin{itemize}
         \item $R_*$: average rate of star formation in our galaxy,
         \item $f_p$: the fraction of formed starts that have planets,
         \item $n_e$: for stars that have planets, the number of planets that
                      can potentially support life,
         \item $f_l$: the fraction of those planets, that actually develop life,
         \item $f_i$: the fraction of planets bearing life on which intelligent,
                      civilized life, has developed,
         \item $f_c$: the fraction of these civilizations that have developed
                      communications, i.e., technologies that release detectable
                      signs into space, and
         \item $L$: the length of time, over which such civilizations release detectable signals.
      \end{itemize}
      \end{footnotesize}
   \end{boxblock}

\end{leftcolumn} %% end left column


%% ------------------------------------------------------------------
%% This \if\else\fi statement is only for the template.
%% ------------------------------------------------------------------
\iflandscape
   %% ---------------------------------------------------------------
   %% Begin center column (for landscape only)
   %% ---------------------------------------------------------------
   \begin{centercolumn}
      %% please leave one blank line here

      \begin{boxblock}{Center Column for Landscape Posters}

         If you ever wondered how the cumulative distribution function of
         the generalized logistic distribution type~I would look like: here it is.
         \begin{equation}
            F(x;\alpha) = \frac{1}{(1 + e^{-x})^\alpha} = (1 + e^{-1})^{-\alpha},~~\alpha > 0
         \end{equation}

         \begin{footnotesize}
         \begin{itemize}
            \item $F(x;\alpha)$: cumulative distribution function
            \item $\alpha$: skewness parameter
         \end{itemize}
         \end{footnotesize}

         \vspace{1em}
         Another equation to fill the space here:
         \begin{equation}
            \frac{\partial}{\partial t}\big(\rho \mathbf{u}\big) + \bigtriangledown \cdot
            \big(\rho \mathbf{u} \otimes \mathbf{u}\big) =
            - \bigtriangledown \cdot p \mathbf{I} + \bigtriangledown \cdot \tau + \rho g
         \end{equation}

         \dots which is the Navier-Stokes equation. If you find an analytic
         solution you might get a quite nice price!
      \end{boxblock}

      \begin{boxblock}{Content Block With Example Figure}

         A very simple statistical graph to demonstrate how the figure
         includes look like in the beamer style and to fill the content
         such that the demo content looks a little bit nicer.

         \begin{figure}
            \centering
            \begin{tikzpicture}[scale=3.5]
                % Draw axes
                \draw [<->,thick] (0,2) node (yaxis) [above] {$y$}
                    |- (3,0) node (xaxis) [right] {$x$};
                % Draw two intersecting lines
                \draw (0,0) coordinate (a_1) -- (2,1.8) coordinate (a_2);
                \draw (0,1.5) coordinate (b_1) -- (2.5,0) coordinate (b_2);
                % Calculate the intersection of the lines a_1 -- a_2 and b_1 -- b_2
                % and store the coordinate in c.
                \coordinate (c) at (intersection of a_1--a_2 and b_1--b_2);
                % Draw lines indicating intersection with y and x axis. Here we use
                % the perpendicular coordinate system
                \draw[dashed] (yaxis |- c) node[left] {$y'$}
                    -| (xaxis -| c) node[below] {$x'$};
                % Draw a dot to indicate intersection point
                \fill[red] (c) circle (2pt);
            \end{tikzpicture}
            \caption{This is just an example figure to demonstrate how
               figure includes with captions look like.}
         \end{figure}

      \end{boxblock}

      \begin{boxblock}{Center Column for Landscape Posters}

         Nam liber tempor cum soluta nobis eleifend option congue nihil imperdiet doming
         id quod mazim placerat facer possim assum. Lorem ipsum dolor sit amet,
         consectetuer adipiscing elit, sed diam nonummy nibh euismod tincidunt ut
         laoreet dolore magna aliquam erat volutpat.

         \vspace{1em}
         Ut wisi enim ad minim veniam, quis
         nostrud exerci tation ullamcorper suscipit lobortis nisl ut aliquip ex ea
         commodo consequat.   
         
         \vspace{1em}
         Nam liber tempor cum soluta nobis eleifend option congue nihil imperdiet doming
         id quod mazim placerat facer possim assum.
         
         \vspace{1em}
         Duis autem vel eum iriure dolor in hendrerit in vulputate velit esse molestie
         consequat, vel illum dolore eu feugiat nulla facilisis. 

      \end{boxblock}

   \end{centercolumn} %% end center column
\fi


%% ------------------------------------------------------------------
%% begin right column for both, portrait and landscape
%% ------------------------------------------------------------------
\begin{rightcolumn}
   %% please leave one blank line here

   \begin{boxblock}{Supported Text Styles and Colors}

      Both, \textbf{bold face} and \textit{italic} styles are supported
      by the poster theme. Beside text styles a set of default colors
      and commands which can be used. These colors are based on the colors of the
      corporate design of our university.

      \vspace{1em}
      \begin{minipage}[t]{.49\textwidth}
         {\bf Available colors:}

         \fcolorbox{black}{uibkblue}{\rule{0pt}{.6em}\rule{.5em}{0pt}}\quad
            blue (\verb|uibkblue|)

         \fcolorbox{black}{uibkbluel}{\rule{0pt}{.6em}\rule{.5em}{0pt}}\quad
            light blue (\verb|uibkbluel|)

         \fcolorbox{black}{uibkorange}{\rule{0pt}{.6em}\rule{.5em}{0pt}}\quad
            orange (\verb|uibkorange|)

         \fcolorbox{black}{uibkorangel}{\rule{0pt}{.6em}\rule{.5em}{0pt}}\quad
            light orange (\verb|uibkorangel|)

         \fcolorbox{black}{uibkgray}{\rule{0pt}{.6em}\rule{.5em}{0pt}}\quad
            gray (\verb|uibkgray|)

         \fcolorbox{black}{uibkgraym}{\rule{0pt}{.6em}\rule{.5em}{0pt}}\quad
            medium gray (\verb|uibkgraym|)

         \fcolorbox{black}{uibkgrayl}{\rule{0pt}{.6em}\rule{.5em}{0pt}}\quad
            light gray (\verb|uibkgrayl|)
      \end{minipage}
      \begin{minipage}[t]{.49\textwidth}
         {\bf Available commands:}

         \begin{table}
         \begin{tabular}{l l}
            \hline
            command & output example \\
            \hline \hline
            \verb|\fct{...}|     &  \fct{example} \\
            \verb|\class{...}|   &  \class{example} \\
            \verb|\pkg{...}|     &  \pkg{example} \\
            \verb|\email{...}|   &  \email{email} \\
            \verb|\doi{...}|     &  \doi{example} \\
            \verb|\file{...}|    &  \file{example} \\
            \verb|\dataset{...}| &  \dataset{example} \\
            \hline
         \end{tabular}
         \caption{Commands provided by the \class{beamerstyleuibk} template.}
         \end{table}


      \end{minipage}

      All commands using verbatim (\verb|\email|, \verb|\doi|, \verb|\file| and \verb|dataset|)
      use a highlight color which can be adjusted by including e.g.,
      \texttt{\setbeamercolor{verbcolor}{fg=\uibkorange}} in the preamble if required.

   \end{boxblock}

   %% Last block
   \begin{boxblock}{Take Home Message}

      Duis autem vel {\bf eum iriure dolor} in hendrerit in vulputate velit esse molestie
      consequat, vel illum dolore eu feugiat {\bf nulla facilisis at vero}.

      \vspace{1em}
      Lorem ipsum dolor sit amet, consectetuer
      adipiscing elit, {\bf sed diam nonummy nibh} euismod tincidunt ut laoreet dolore
      magna aliquam erat {\bf volutpat}.   

      \vspace{1em}
      \begin{itemize}
         \item Duis autem vel eum iriure
         \item dolor in hendrerit in vulputate, vel illum dolore
         \item velit esse molestie consequat
      \end{itemize}
      
      \vspace{1em}
      Ut wisi enim {\bf ad minim veniam}, quis nostrud exerci tation ullamcorper suscipit
      lobortis nisl ut aliquip ex ea commodo consequat.
      
   \end{boxblock}

   %% References
   \begin{footnotesize}
   
   \vspace{0.3cm}
   \begin{minipage}[t]{0.75\textwidth}
      \textbf{References:} \\
      %\bibliographystyle{ametsoc}
      %\bibliography{EMS}
      Mustermann, M. and Demoman F.C., 2017:
        A Fake Reference to Demonstrate How This Could Look like.
        LaTeX poster template demo, \textbf{0}(0), 666-999.
      \vspace{1cm}
   
      \textbf{Acknowledgements:} \\
      Ongoing project funded by the Austrian Science Fund (FWF): TRP 123-456.
      The computational results presented have been achieved (in part) using
      the HPC infrastructure LEO of the University of Innsbruck.
   \end{minipage}
   \hfill
   \begin{minipage}[t]{0.12\textwidth}
      \begin{figure}
         \includegraphics[width=\textwidth]{license_ccby}
         \vspace{5mm}
   
         \includegraphics[width=\textwidth]{qrcode} \\
      \end{figure}
   \end{minipage}
   \end{footnotesize}
\end{rightcolumn}
%% end right column %%%%%%%%%%%%%%%%%%%%%%%%%%%%%%%%%%%%%%%%%%%%%%%%%%
%%%%%%%%%%%%%%%%%%%%%%%%%%%%%%%%%%%%%%%%%%%%%%%%%%%%%%%%%%%%%%%%%%%%%%
%%%%%%%%%%%%%%%%%%%%%%%%%%%%%%%%%%%%%%%%%%%%%%%%%%%%%%%%%%%%%%%%%%%%%%
%%%%%%%%%%%%%%%%%%%%%%%%%%%%%%%%%%%%%%%%%%%%%%%%%%%%%%%%%%%%%%%%%%%%%%

\end{columns}
\end{frame}

\end{document}
